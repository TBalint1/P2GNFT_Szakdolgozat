\Chapter{Bevezetés}

Manapság a sport hatalmas szerepet játszik a világ és az emberek életében. A legnépszerűbb 
sportágak, mint a labdarúgás, a kosárlabda, a jégkorong és az amerikai futball rendkívül 
nagy befolyással vannak a társadalomra, a gazdaságra, a kultúrára, a politikára, a médiára és 
szinte a teljes hétköznapi életre. A népszerűbb sportágaknál a növekedés jelenlegi fázisa hogy, a fejlődő országokba is teret hódíthassanak azon országokhoz hasonlóan ahol 
már jelenleg is nagy népszerűségnek örvendenek és befolyásoló hatással vannak a 
társadalomra.

Ezzel párhuzamosan növekednek és hódítanak teret a „kisebb”, kevésbé népszerű sportágak is, természetesen más formában. Ezen sportágak közé tartozik a darts is. Habár a darts a szülőországában, Angliában már az 1920-as évek óta nagy népszerűségnek örvend, ugyanis ekkor 
kezdődött el a sportág standardizálása és a szabályok egységesítése – de az utóbbi két évtizedben már több országban is hasonló népszerűséget és sikereket ért el, mint például Hollandiában, Skóciában, Németországban, Walesben, Belgiumban, 
Ausztráliában és az Egyesült Államokban. A felsorolt országok közül az angolok mellett nagy sikereket értek el a skót, a walesi és a holland játékosok, de a többi ország játékosai is a feltörekvők közé sorolhatók, minden bizonnyal már csak idő kérdése és hasonló sikereket fognak elérni, mint a korábban említett társaik. 

A teljesítménytől eltérően már sokkal nehezebb kimutatni a különbséget a felsorolt országok között a sportág népszerűsége terén, hiszen az ezekben az országokban megrendezett versenyek minden egyes alkalommal rendkívül magas számú nézőt vonzanak és ezzel a párhuzamosan magasabb a versenyek színvonala, komolysága az átlagosnál, de azt is meg kell említeni, hogy ezekben az országokban rendezik a szezon egyik legnépszerűbb versenyének, a Premier League Darts-nak a fordulóit is. A népszerűség kapcsán még fontos megemlíteni, hogy a minden évben megrendezendő világbajnokságnak is ezekben az országokban vannak a legmagasabb nézőszámai. „Hollandiában az RTL7-nek a 2013-as és 2014 döntő, amelyben Michael Van Gerwen játszott, 1,7 és 2,05 milliós nézettséget hozott, míg 2016-ban Van Barneveld és honfitársa, Klassen összecsapása hozott 1,56 milliós nézettséget. Ugyanakkor a rekordot a 2017-es döntő tartja a maga 2,07 milliós nézettségével, melyben Van Gerwen második világbajnoki 
győzelmét aratta. Németországban a Sport1 csatorna közvetítését a 2017-ös döntő alkalmával követték a legtöbben, 1,48 millióan, míg a 2015-ös Angliában rekord nézettséget hozó döntőt 1,36 millióan követték figyelemmel.” –olvasható a sportmarketing.hu cikke alapján, amelyet a darts sikerének titkáról írtak 2017-ben.
\cite{Darts}


A darts sikerével együtt járnak természetesen a támogatók is, amelyeknek a legnagyobb részét a sportfogadási szervezetek teszik ki. Ebből adódóan elengedhetetlen a statisztikák részletes és megfelelő feldolgozása is, amely habár rengeteget növekedett az elmúlt években, de még mindig fejlesztésre szorul. Az utóbbi tény hatványozottan igaz azon országokra amelyekben, még nem vált olyan meghatározó és népszerű sporttá a darts, ilyen például Magyarország is, ahol sem statisztikák, sem sportfogadás terén nem áll rendelkezésre annyi lehetőség, mint a népszerűbb sportoknál. Ez az adatok feldolgozásának hiányából eredeztethető. Ugyan a nagyobb versenyeknél találunk statisztikákat Magyar weboldalakon, például az eredmények.com–on (a nemzetközi sportstatiszikai oldal a flashscore.com magyar megfelelője, de ez lényegesen kevesebb adat a külföldi országokban működő oldalakhoz képest, és emellett gyakran előfordul, hogy a kisebb, de nemzetközi versenyekről semmilyen statisztikát nem találunk az eredményen kívül. Szembetűnő ez a különbség a sportfogadási oldalaknál is, például a Magyar Szerencsejáték Zrt.–nél lényegesen kevesebb fogadási lehetőségünk van mint a nemzetközi sportfogadási oldalaknál, mint például az Unibet, Bet365, stb. Ezt a különbséget is az adatok feldolgozásának hiánya eredményezi. Mint ahogy a nagyobb külföldi oldalaknál láthatjuk, az adatok feldolgozásának és az abból való részletes statisztikák előállításának a lehetősége megvan, és mivel hazánkban az elmúlt években kezd jelentősen népszerűbbé válni a sportág, ezért az igény is meg lenne rá, ennek ellenére ez egyenlőre még egy úgynevezett piaci rés Magyarországon.

A webalkalmazásom célja a részletesebb statisztikák előállításának bemutatása, a felhasználók és általam lejátszott, szimulált mérkőzéseken keresztül. Ezt saját versenyek, mérkőzések létrehozásával teszem lehetővé. A létrejött mérkőzések lejátszásával párhuzamosan kerülnek kiszámításra az adott mérkőzés statisztikái, ezek a mérkőzések és statisztikák pedig egy adatbázisban kerülnek tárolásra, ahonnan az adatok lekérdezhetők és megjeleníthetők.
