\Chapter{Implementáció}
\Section{Oldalak kapcsolódása}
Ebben a pontban szeretném szemléltetni az alkalmazás egyes oldalai közötti kapcsolódást. Az olyan adatmodelleknél, mint a mérkőzések, a versenyek és a játékosok, az oldalakat két részre osztjuk, ugyanis egyes oldalakon többet is el tudunk érni közülük, míg az adatlapjaik az azonosítójuk segítségével érhető el.
\begin{cpp}
const routes: Routes = [
  { path: '', component: HomeComponent },
  { path: 'search/:searchTerm', component: AllPlayersComponent },
  { path: 'players', component: AllPlayersComponent },
  { path: 'players/:id', component: PlayerPageComponent },
  { path: 'tournaments', component: TournamentsComponent },
  { path: 'matches/:id', component: MatchPageComponent },
  { path: 'login', component: LoginPageComponent },
  { path: 'register', component: RegisterPageComponent },
  { path: 'tournaments/new', component: NewTournamentPageComponent },
  { path: 'tournament/:id', component: TournamentPageComponent },
  { path: 'matches/:id/onGoing', component: DartsPartyPageComponent },
];
\end{cpp}
Az itt definiált útvonalak határozzák meg, hogy az alkalmazás komponensei, oldalai, milyen címen érhetőek el az alkalmazásból.
\Section{Versenyek létrehozása}
Egy versenyt a regisztrációhoz hasonlóan egy űrlap kitöltésével hozhatunk létre. Miután kitöltöttük az űrlapot, a backend segítségével az itt megadott adatokból jön létre az új verseny az adatbázisban, azonban a létrehozásával, nem csupán egy versenyt hozunk létre, hanem létrejönnek a versenyeken résztvevő játékosok is és elkészül az első forduló sorsolása, vagyis a létrejönnek az első mérkőzések. Ez azért lehetséges, mivel a játékosok és a mérkőzések is csupa olyan rekorddal rendelkeznek, amelyek a versenyből hivatkozhatók. Így a játékosok és a mérkőzések szükséges rekordjait beállítjuk a verseny megfelelő rekordjának az értékére, az azonosítókat pedig a MongoDB hozza létre.

A versenyt egy Post kéréssel tudjuk létrehozni ami a következőképpen néz ki:
\begin{itemize}
\item Az első lépésben a kért adatok alapján létrejön maga a verseny.
\begin{cpp}
router.post(
 "/new",
 asyncHandler(async (req, res) => {
  const {
   name, 
   type, 
   playersCount, 
   players, 
   points, 
   legs, 
   doubleOut
  } = req.body;

  const existingPlayer = await PlayerModel.findOne({ players });
    if (existingPlayer) {
      res
        .status(HTTP_BAD_REQUEST)
        .send("Player with this name already exists!");
      return;
    }

  const newTournament: Tournament = {
    id: "",
    name: req.body.name,
    type: req.body.type,
    playersCount: req.body.playersCount,
    points: req.body.points,
    legs: req.body.legs,
    doubleOut: req.body.doubleOut,
    players: req.body.players,
    currentRound: "Round 1",
    winner: "",
    runnerUp: "",
  };

  const dbTournament = await TournamentModel.create(newTournament);
  
  //...
  
 }
);
\end{cpp}
\item Ezt követően a verseny felvétele alatt megadott számú játékos számára is létrejönnek a \textit{Player} modell dokumentumok az adatbázisban. Ehhez a játékosok nevének a megadása szükséges. 
\begin{cpp}
let dbPlayers: Player[] = [];

for (let i = 0; i < newTournament.playersCount; i++) {
  const newPlayer: Player = {
    id: "",
    name: newTournament.players[i].playerName,
    tournament_win: 0,
    tournament_lose: 0,
    match_win: 0,
    match_lose: 0,
   };

    const dbPlayer = await PlayerModel.create(newPlayer);
    dbPlayers.push(dbPlayer);
}
\end{cpp}
\item A következő lépésben hajtódik végre a sorsolás, vagyis a játékosokkal kettesével generál mérkőzéseket, amelyek később majd leszimulálhatók lesznek.
\begin{cpp}
let dbMatches: Match[] = [];

for (let i = 0; i < playersCount; i += 2) {
  const newMatch = new Match(
    "",
    dbTournament.id,
    "Round 1",
    req.body.legs,
    req.body.doubleOut,
    req.body.points,
    "",
    dbPlayers[i].id,
    dbPlayers[i].name,
    0,
    dbPlayers[i + 1].id,
    dbPlayers[i + 1].name,
    0,
    []
   );

   const dbMatch = await NewMatchModel.create(newMatch);
   dbMatches.push(dbMatch);
}
\end{cpp}
\item Végül az utolsó lépésben csoportosítjuk az adatokat egy objektumba, majd pedig ezt az objektumot küldjük el válaszként.
\begin{cpp}
const Data = {
   tournament: dbTournament,
   players: dbPlayers,
   match: dbMatches,
};
res.send(Data);
\end{cpp}
\end{itemize}
\Section{Pontkalkulátor}
A pontkalkulátor segítségével tudjuk a létrehozott mérkőzéseket lejátszani. A feladata, hogy az inputról kapott adatokat, vagyis a dobásokat, jól fel tudja dolgozni. Ennek érdekében egy \textit{DartsParty} nevű osztályt hoztam létre. Az első metódusa a mérkőzés elindításáért felelős, a második metódus dolgozza fel, továbbítja és frissíti az adatokat a mérkőzésben.

Első lépésben meg kell hívnunk a \texttt{start()} metódust, amely betölti az első üres leget a mérkőzésbe.
\begin{cpp}
  start(): Match {
    this.match.legs.push(this.currentLeg);
    return this.match;
  }
\end{cpp}

A metódust egy Put kéréssel tudjuk meghívni. Amikor ez megtörténik, létrejön az adatbázisban két dokumentum a játékosok statisztikáinak a tárolására, majd pedig ezek az adatok a memóriába kerülnek, ugyanis, a későbbiekben a mérkőzés lejátszásához szükséges függvény is csak akkor fog helyesen lefordulni, ha az összes kért adat megtalálható a memóriában. Ezt követően hívjuk meg a \texttt{start()} metódust az eltárolt mérkőzésre és miután ez végrehajtódott, az adatokat elküldjük az adatbázis számára. Így létrejön az adatbázisban a két statisztikai adattábla és frissül a mérkőzés az új leggel.

\begin{cpp}
router.put(
  "/:matchID/onGoing",
  asyncHandler(async (req, res) => {
    const match = await NewMatchModel.findById(req.params.matchID);
    let dbStats: Stat[] = [];

    if (match === null) {
      res.send("Null response!");
    } else {
      const homeStat = new Stat(
        "",
        match.tournamentId,
        match.id,
        match.homeId,
        0, 0, 0, 0, 0, 0, 0, 0, 0, 0, 0, 0, 0, 0, 0
      );

      const dbHomeStat = await StatModel.create(homeStat);
      dbStats.push(dbHomeStat);

      const awayStat = new Stat(
        "",
        match.tournamentId,
        match.id,
        match.awayId,
        0, 0, 0, 0, 0, 0, 0, 0, 0, 0, 0, 0, 0, 0, 0
      );

      const dbAwayStat = await StatModel.create(awayStat);
      dbStats.push(dbAwayStat);
      
      dartsParty = new DartsParty(match, homeStat, awayStat);
    }
    
    console.log(match);
    try {
      const matchID = req.params.matchID;

      const updatedMatch = await dartsParty.start();

      const data = {
        match: updatedMatch,
        stats: dbStats,
      };

      await NewMatchModel.findByIdAndUpdate(matchID, updatedMatch);
      res.send(data);
    } catch (error) {
      res.status(500).send("Update Failed");
    }
  })
);
\end{cpp}

Miután elindult a mérkőzés a két játékos elkezdheti a dobásai rögzítését felváltva. Ezt a \texttt{throw(turn: Turn)} metódussal tudjuk megtenni, ami egy \textit{Turn} objektumot vár bemenetnek, ami jelen esetben egy játékos dobását jelenti egy körben, végezetül pedig majd a mérkőzéssel tér vissza. Miután meghívtuk, a függvény a kapott \textit{Turn} objektumból kiszámítja az első, a második és a harmadik nyíl értékét lokális változókba és a \textit{Turn} objektumot hozzáadja a \texttt{currentLeg}-hez.

\begin{cpp}
thrown(turn: Turn): Match | undefined {
 if (
  this.currentLeg.turns.length === 0 && 
  this.currentLeg.starterPlayer !== turn.playerId
 ) {
  return undefined;
 }
 const dart1 = turn.throw1Sector * turn.throw1Multiplier;
 const dart2 = turn.throw2Sector * turn.throw2Multiplier;
 const dart3 = turn.throw3Sector * turn.throw3Multiplier;
 const previousTurn = 
  this.currentLeg.turns?.[this.currentLeg.turns.length - 1]?.playerId;
 if (previousTurn === undefined || previousTurn !== turn.playerId) {
  let currentPlayer = turn.playerId;
  this.currentLeg.turns.push(turn);
  
  \\...
  
 }
 
 \\...
 
}
\end{cpp}

Ezt követően két ágra bontjuk a számításokat azalapján, hogy a mérkőzésen duplával kell-e kiszállni vagy sem, azonban ez csak a leg befejezését befolyásolja, a többi adat ettől függetlenül ugyanúgy kerül kiszámításra minden ágon.

Miután beléptünk az egyik ágba első sorban ellenőrizzük, hogy melyik volt a dobó játékos és annak az adatait módosítjuk. A játékos pontjait a \texttt{currentPoint} lokális változóban tartjuk számon, amely minden nyíl után módosításra kerül. Először az első nyíl értékét vonjuk le a változóból és ellenőrizzük, hogy ezzel a dobással kiszállt-e a játékos vagyis, hogy a pontjainak az értéke pontosan nulla-e, valamint dupla kiszálló estében dupla szektor, volt-e az eltalált szektor. 

Amennyiben a játékos kiszállt beállítjuk az azonosítóját a leg győztesének és levonjuk a \texttt{currentPoint}-ot a \texttt{currentLeg.homePoint} vagy \texttt{currentLeg.awayPoint} változóból, attól függően hogy melyik játékos dobott. 

Ezután kerül meghívásra a \texttt{startNewLegIfPossible()} függvény, amelynek a feladata az, hogy ellenőrizze, hogy ezzel a játékos elérte-e a mérkőzés megnyeréséhez szükséges legek számát, vagyis megnyerte-e a mérkőzést, ha viszont még nem nyerte meg akkor elindítunk egy új leget.

\begin{cpp}
startNewLegIfpossible() {
 if (this.match.homeScore === this.match.firstTo) {
  this.match.winner = this.match.homeId;
 } else if (this.match.awayScore === this.match.firstTo) {
    this.match.winner = this.match.awayId;
 } else {
   if (this.servicePlayer === this.match.homeId) {
    this.servicePlayer = this.match.awayId;
   } else {
     this.servicePlayer = this.match.homeId;
   }
   this.match.legs.push(this.currentLeg);
   this.currentLeg = new Leg(
     this.servicePlayer,
     "",
     this.match.points,
     this.match.points,
     []
    );
    this.match.legs.push(this.currentLeg);
  }
}
\end{cpp}

Abban az esetben, ha a játékos még nem szállt ki az első nyíl eldobása után, akkor levonjuk a második eldobott nyíl értékét és ugyanazokat az ellenőrző lépéseket hajtjuk végre, mint az első nyíl esetében. Ezt követi a harmadik nyíl is, és amennyiben a játékosnak egyik nyíllal se sikerült kiszállnia, vagyis a körben dobott nyilainak az értéke kevesebb mint a játékos pontszáma, akkor csak egyszerűen levonásra kerül a dobás a játékos pontszámából, ha viszont a dobás értéke több mint a játékos pontszáma, akkor a játékos pontszáma nem változik mivel "besokalt", vagyis a dobás értéke 0. Az utolsó lépésben a megváltoztatjuk a dobó játékost a következő játékosra, frissítjük a leget és a mérkőzéssel térünk vissza.

\begin{cpp}
if (this.match.doubleOut === true) {
  if (currentPlayer === this.match.homeId) {
    this.previousHomePoint = this.currentLeg.homePoint;
    const currentPoint = this.currentLeg.homePoint - dart1;
    if (currentPoint === 0 && turn.throw1Multiplier === 2) {
      this.currentLeg.winner = this.match.homeId;
      this.currentLeg.homePoint = currentPoint;
      this.match.homeScore++;
      this.startNewLegIfpossible();
    } else if (currentPoint > 0) {
      const currenPoint = currentPoint - dart2;
      if (currenPoint === 0 && turn.throw2Multiplier === 2){
       this.currentLeg.winner = this.match.homeId;
       this.currentLeg.homePoint = currentPoint;
       this.match.homeScore++;
       this.startNewLegIfpossible();
      } else if (currenPoint > 0) {
        const currentPoint = currenPoint - dart3;
        if (currentPoint === 0 && turn.throw3Multiplier === 2) {
          this.currentLeg.winner = this.match.homeId;
          this.currentLeg.homePoint = currentPoint;
          this.match.homeScore++;
          this.startNewLegIfpossible();
        } else if (currentPoint > 0) {
          this.currentLeg.homePoint = currentPoint;
        }
      }
    }
    currentPlayer = this.match.awayId;
  } else {
    this.previousAwayPoint = this.currentLeg.awayPoint;
    const currentPoint = this.currentLeg.awayPoint - dart1;
    
    \\...
    
    }
}
\end{cpp}

A \texttt{throw(turn: Turn)} metódust szintén egy Put kéréssel tudjuk meghívni. Mivel a mérkőzés már el van tárolva a memóriában, így nem szükséges azt újra lekérdezni az adatbázisból, csak definiálunk egy \textit{Turn} objektumot amiben az adatokat továbbítani fogjuk. Ezután meghívjuk a metódust a \texttt{turn}-re, és ezt a \texttt{state} változóban tároljuk, ami a visszatérési értékből eredően egy \textit{Match} objektum lesz. Itt történik a statisztikák kiszámítása is a \texttt{calculateStatistics(turn: Turn)} metódus meghívásával amelyet majd a következő pontban fogok kifejteni.

Ezt követően megkeressük az adatbázisban az adott mérkőzést és statisztikát, majd pedig frissítjük azokat a két függvény visszatérési értékével. Az adatokat egy \texttt{data} objektumba gyűjtjük össze és ezt az objektumot küldjük válaszul.

\begin{cpp}
router.put(
  "/:matchID/onGoing/throw",
  asyncHandler(async (req, res) => {
    const {
      playerID,
      throw1Sector,
      throw1Multiplier,
      throw2Sector,
      throw2Multiplier,
      throw3Sector,
      throw3Multiplier,
    } = req.body;
    const turn = new Turn(
      req.body.playerID,
      req.body.throw1Sector,
      req.body.throw1Multiplier,
      req.body.throw2Sector,
      req.body.throw2Multiplier,
      req.body.throw3Sector,
      req.body.throw3Multiplier
    );
    const state = dartsParty.thrown(turn);
    const statistic = dartsParty.calculateStatistics(turn);
    if (state !== undefined) {
      await NewMatchModel.findByIdAndUpdate(req.params.matchID, state);
      await StatModel.findOneAndUpdate(
        { playerId: req.body.playerID, matchId: req.params.matchID },
        statistic
      );
      const data = {
        match: state,
        statistic: statistic,
      };
      res.send(data);
    } else {
      res.send("Invalid turn!");
    }
  })
);
\end{cpp}

\Section{Statisztikák feldolgozása}
Ebben a pontban szeretném bemutatni a statisztikák kiszámításának a megvalósítását. Ahogyan azt az előző pontban is említettem, ezt a \texttt{calculateStatistics(turn: Turn)} függvény hajt végre, amit szintén a DartsParty osztályon belül található meg, bemeneti értéknek szintén egy \textit{Turn} objektumot vár a  \texttt{throw} függvényhez hasonlóan, és egy  \textit{Stat} objektummal tér vissza, ami egy mérkőzésen az adott játékos statisztikáit tárolja.

A függvény meghívásakor kiszámításra kerül a 3 dobott nyíl értéke külön külön és egyben is a \texttt{thrownPoints} változóban. Ezután két ágra bomlik a metódus attól függően, hogy melyik játékos volt az adott kör dobó játékosa. Miután beléptünk a két ág egyikére első lépésben módosítjuk azon lokális változókat, amelyek csak úgynevezett "segéd" változók a statisztikák kiszámításához, mint például az egyik játékos által dobott értékek összege, a dobott köreinek a száma, stb.

Ezután sorban haladunk a különböző statisztikai szempontokon és implementáljuk azokat a statisztikai számításokat amelyeket már a 3. fejezet 2. pontjában kifejtettem. 

\begin{cpp}
calculateStatistics(turn: Turn): Stat {
  const dart1 = turn.throw1Sector * turn.throw1Multiplier;
  const dart2 = turn.throw2Sector * turn.throw2Multiplier;
  const dart3 = turn.throw3Sector * turn.throw3Multiplier;
  const thrownPoint = dart1 + dart2 + dart3;
  if (turn.playerId === this.match.homeId) {
   this.homeTurnCount++;
   this.homeSumPoint += thrownPoint;
   this.homeDart1Sum += dart1;
   this.homeDart2Sum += dart2;
   this.homeDart3Sum += dart3;
      
   this.homeStat.average = this.homeSumPoint / this.homeTurnCount;
  }
  
  \\...
}      
\end{cpp}

A statisztikai szempontok közül szeretném kiemelni a kiszálló hatékonyság kiszámítását, mivel ez volt a legösszetettebb. 

Elsősorban ellenőrizzük, hogy a játékos pontszáma kiszállónak számít-e és hány nyíl szükséges hozzá. Ezekhez 3 tömböt hoztam létre az osztály elején amelyek tartalmazzák az 1, 2 vagy 3 nyilas kiszállókat. Amennyiben a játékos pontszáma kiszállónak számít, akkor a függvény növeli az adott játékos kiszállási lehetőségeinek a számát.
Ezután ellenőrzi, hogy a játékos kiszállt-e, vagyis megegyezik-e a dobott pontszáma a kör előtti pontszámával. Amennyiben a játékos kiszállt, akkor növeli a sikeres kiszállóinak a számát és kiszámítja a sikeres kiszállók arányát, a sikeres kiszállók és a lehetséges kiszállási lehetőségek osztásával. A következő lépésben a függvény ellenőrzi, hogy a dobott kiszálló értéke meghaladja-e az eddigi legmagasabb kiszállónak az értékét, amennyiben pedig igen, akkor a jelenlegi kiszállóra módosítja az értéket, ha viszont nem, akkor változatlanul hagyja. Az utolsó lépésben a függvény ellenőrzi, hogy a dobott kiszálló 1,2, vagy 3 nyilas kiszállónak számít-e, és amennyiben az egyik ágra belép, akkor növeli annak a típusú kiszállóknak a számát.

\begin{cpp}      
if (
 possible3DartsCheckouts.includes(this.previousHomePoint) || 
 possible2DartsCheckouts.includes(this.previousHomePoint) || 
 possible1DartCheckout.includes(this.previousHomePoint)
) {
 this.homePossibleCheckoutCount++;
        
 if (this.previousHomePoint === thrownPoint) {
  this.homeCheckoutCount++;
  this.homeStat.checkouts = this.homeCheckoutCount / this.homePossibleCheckoutCount;
  if (thrownPoint > this.homeStat.highestCheckout) {
    this.homeStat.highestCheckout = thrownPoint;
  }
  if (possible3DartsCheckouts.includes(thrownPoint)) {
    this.homeStat.numberOf3DartCheckouts++;
  } else if (possible2DartsCheckouts.includes(thrownPoint)) {
    this.homeStat.numberOf2DartCheckouts++;
  } else if (possible1DartCheckout.includes(thrownPoint)) {
    this.homeStat.numberOf1DartCheckouts++;
  
  }
 }
}      
\end{cpp}

Innen folytatódik a metódus a 180-as, a 140 feletti, és a 100 feletti dobások megszámolásával.

\begin{cpp}
if (thrownPoint === 180) {
 this.homeStat.numberOf180s++;
}
if (thrownPoint >= 140 && thrownPoint < 180) {
 this.homeStat.numberOf140plus++;
}
if (thrownPoint >= 100 && thrownPoint < 140) {
 this.homeStat.numberOf100plus++;
}
\end{cpp}

Ezt követően kerül kiszámításra az első 9 nyíl átlaga, valamint a körönként eldobott első, második, és harmadik nyíl értékének az átlaga.

\begin{cpp}
if (this.homeTurnCount < 4) {
 this.homeStat.first9DartsAverage = this.homeSumPoint / this.homeTurnCount;
}
this.homeStat.firstDartAvergrage = this.homeDart1Sum / this.homeTurnCount;
this.homeStat.secondDartAverage = this.homeDart2Sum / this.homeTurnCount;
this.homeStat.thirdDartAverage = this.homeDart3Sum / this.homeTurnCount;
\end{cpp}

Az utolsó pontokban megszámoljuk azon nyilak számát, amelyekkel a játékos a tripla 20-as szektor találta el, valamint elosztjuk a játékos 180-as dobásainak a számát az általa eldobott körök számával, amiből megkapjuk a körönként dobott 180-as dobások arányát. Végezetül pedig a az adott játékos statisztikájával térünk vissza.

\begin{cpp}
if (dart1 === 60) {
 this.homeStat.triple20s++;
}
if (dart2 === 60) {
 this.homeStat.triple20s++;
}
if (dart3 === 60) {
 this.homeStat.triple20s++;
}

this.homeStat.percentageOf180PerLeg = 
 this.homeStat.numberOf180s / this.homeTurnCount;
        
return this.homeStat;
\end{cpp}