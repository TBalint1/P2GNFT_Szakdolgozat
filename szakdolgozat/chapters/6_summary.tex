\Chapter{Összefoglalás}

A dolgozatom célja az volt, hogy bemutassa a statisztikák feldolgozását a darts esetében, és ehhez szemléltetés - megvalósításképpen választottam a webalkalmazás fejlesztést. Habár a darts számomra már felfedezett terület volt, még így is rengeteg új dolgot tanultam meg vele kapcsolatban a dolgozat írása közben, ezért élveztem a kutatómunkát. Hasonlóképpen volt ez a webalkalmazás fejlesztésének megvalósításával is, amiben ugyan nem volt sok tapasztalatom, ezért gyakran ütköztem nehézségekbe. Ezekben a problémákban segítséget nyújtottak az interneten megtalálható dokumentációk, amiből szerencsére több is fellelhető, az általam választott programozási nyelvek népszerűségének köszönhetően.

A dolgozatommal szerettem volna bemutatni a JavaScript keretrendszerű programnyelvek előnyeit és, hogy hogyan alkalmazhatók webalkalmazás fejlesztésére. A fejlesztés során rengeteg olyan funkciójával és tulajdonságával találkoztam mind az Angularnak, mind a Node.js-nek, amelyek megkönnyítették a munkámat. A JavaScript keretrendszernek köszönhetően, nem voltak kompatibilitásbeli problémáim, így könnyebb volt az alkalmazás Backend és Frontend oldalának az összekötése. 

Korábban volt már tapasztalatom alkalmazások tervezésével, de a dolgozat írásával betekintést nyerhettem a fejlesztői feladatkörbe is. A részletes tervezéssel leegyszerűsítettem a teendőimet, majd pedig a tesztelési fázisban sokat tudtam javítani a programon ezáltal a végső fázisba hozva azt. Ezeket a pontokat igyekeztem a dolgozatomban bemutatni, hogy minél realisztikusabb képet tudjak adni a webfejlesztésről.

A programom ugyan még funkciók és minőség terén elmarad a piacon lévő hasonló funkciójú programoktól, de érdekes volt számomra, hogy a fejlesztés által megtapasztalhattam olyan programok működését, amelyeket én is sűrűn használok dartsozás közben.

Az alkalmazásban akadnak még, optimalizálási és fejlesztési lehetőségek, ezek közül pedig kiemelném az alkalmazás reszponzívvá tételét, hiszen egy ilyen alkalmazás esetében hasznos, ha a felhasználó több készülékről érheti el és ezáltal szinte bárhol alkalmazhatja azt. Természetesen az alkalmazás bővíthető még további játékmódokkal, és játékbeli funkciókkal is.

Mivel a tanulmányaim elvégzése után nem szeretnék eltávolodni sem a programozástól, sem a dartstól, ezért ezekkel a fejlesztési lehetőségekkel a jövőben bővíteni szeretném az alkalmazásomat. 
