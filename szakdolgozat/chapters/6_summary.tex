\Chapter{Összefoglalás}

A dolgozatom célja az, volt hogy bemutassa a statisztikák feldolgozását a darts esetében és ehhez szemléltetés, megvalósításképpen választottam a webalkalmazás fejlesztését. Habár a darts számomra már felfedezett terület volt, még így is rengeteg új dolgot tanultam meg vele kapcsolatban a dolgozat írása közben, ezért élveztem is a kutatómunkát. Hasonlóképpen volt ez a webalkalmazás fejlesztésének megvalósításával is, amiben ugyan nem volt sok tapasztalatom, ezért gyakran ütköztem nehézségekbe, de az ezzel kapcsolatos kutatómunkák során könnyen megoldottam ezeket a nehézségeket, hiszen olyan sok dokumentáció található meg az interneten, hogy sokszor azt se egyszerű eldönteni, hogy melyik megoldást válasszuk a problémára.

A dolgozatommal szerettem volna bemutatni a JavaScript keretrendszerű programnyelvek előnyeit és, hogy hogyan alkalmazhatók webalkalmazás fejlesztésére. Az fejlesztés során rengeteg olyan funkciójával és tulajdonságával találkoztam mind, az Angularnak, mind a Node.js-nek, amivel megkönnyítették, lerövidítették a munkámat és JavaScript keretrendszernek köszönhetően, nem voltak kompatibilitásbeli problémáim így könnyebb volt az alkalmazás backend és frontend oldalának az összekötése. 

A dolgozat írásával betekintést nyerhettem az informatika egy olyan részébe, ami eddig még soha és ez pedig egy teljes alkalmazás fejlesztés volt. Megtapasztalhattam, hogy milyen fontos fázisai vannak egy teljes fejlesztésnek a program írásán kívül is. A részletes tervezésekkel sokat könnyítettem a feladataimon, majd pedig a tesztelési fázisban sokat tudtam javítani a programon ezáltal a végső fázisba hozva azt. Ezeket a pontokat igyekeztem a dolgozatomban bemutatni, hogy minél realisztikusabb képet tudjak adni a webfejlesztésről.

A programom ugyan még funkciók és minőség terén elmarad a piacon lévő hasonló funkciójú programoktól, de érdekes volt számomra, hogy a fejlesztés által betekintést nyerhettem olyan programok mögé amelyeket én is sűrűn használok dartsozás közben.

Az alkalmazásban akadnék még optimalizálási és fejlesztési lehetőségek, ezek közül pedig kiemelném az alkalmazás reszponzívvá tételét, hiszen egy ilyen alkalmazás esetében hasznos, ha a felhasználó több készülékről elérheti és ezáltal szinte bárhol alkalmazhatja. Természetesen az alkalmazás bővíthető még további játékmódokkal és játékbeli funkciókkal is.

Mivel a tanulmányaim elvégzése után nem szeretnék eltávolodni sem a programozástól, sem a Dartstól, ezért ezekkel a fejlesztési lehetőségekkel a jövőben bővíteni szeretném az alkalmazásomat. 
