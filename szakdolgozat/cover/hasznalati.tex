\pagestyle{empty}

\noindent \textbf{\Large A mellékelt CD tartalma}

\vskip 1cm

A szakdolgozatomhoz mellékelt CD két mappát tartalmaz:

\begin{itemize}
\item A \texttt{P2GNFT_Szakdolgozat} nevű mappában található meg a dolgozat forráskódja és a dolgozat PDF formátumban dolgozat.pdf néven.
\item Az elkészített programot a \texttt{P2GNFT_DoubleOut} nevű mappa tartalmazza.
\end{itemize}

A program indítása Visual Studio Code-ból lehetséges és a Node.js telepítése szükséges hozzá.

\begin{itemize}
\item Nyissuk meg a program mappáját Visual Studio Code-ban és nyissunk meg egy új terminált,
\item A terminált navigáljuk a \texttt{backend} mappába a \texttt{cd backend} parancs beírásával,
\item Az \texttt{npm install express cors} parancs beírásával telepítsük az express-t,
\item Ezután indítsuk el a szervert az \texttt{npm start} paranccsal,
\item A következő lépésben nyissunk meg egy új terminált és navigáljunk a \texttt{frontend} mappába a \texttt{cd frontend} parancs beírásával,
\item Miután beléptünk a mappába az \texttt{ng serve -o} parancs beírásával indítsuk el a programot. 
\end{itemize}

Ezen lépések követésével a program a böngészőben fog elindulni http://localhost:4200/ címen.